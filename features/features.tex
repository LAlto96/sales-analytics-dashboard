\documentclass{article}
\usepackage[utf8]{inputenc}
\usepackage{hyperref}
\usepackage{geometry}
\geometry{a4paper, margin=1in}
\usepackage{lipsum} % For dummy text
\hypersetup{
    colorlinks=true,
    linkcolor=black,
    urlcolor=blue,
    }


\title{Sales Dashboard Documentation}
\author{Ludovic Lafon}
\date{\today}

\begin{document}

\maketitle
\tableofcontents
\newpage % Add a new page after the table of contents

\section{Introduction}

\subsection{Purpose of the Document}
This document provides comprehensive documentation of the sales dashboards developed using Tableau. It is intended for sales managers, business executives, and analysts who will be using these dashboards to monitor and analyze sales performance.

\subsection{Overview of the Sales Dashboard}
The sales dashboards include the Homepage Dashboard, Revenue Dashboard, Product Performance Dashboard, Sales Team Dashboard, and Pipeline Analysis Dashboard. Each dashboard is designed to provide insights into different aspects of the sales process, from overall revenue to individual sales agent performance.

\subsection{Importance of Sales Dashboards}
Sales dashboards are crucial for making data-driven decisions. They allow for real-time tracking of key performance indicators (KPIs), identification of trends and patterns, and optimization of sales strategies.

\subsection{Data Sources}
The dashboards integrate data from a fictional company's CRM system, which is available on Kaggle. The dataset includes information about customer interactions, sales activities, and opportunities. The data was cleaned, transformed, and integrated using Pandas in a separate Python notebook file, which is  \href{[https://github.com/LAlto96/sales-analytics-dashboard]}{available on GitHub}.

\subsubsection{About the Dataset}
\textbf{Description:}
This dataset contains information about customer interactions, sales activities, and opportunities from a fictional company's CRM (Customer Relationship Management) system. The dataset is designed to help data scientists and analysts understand the sales process, identify trends and patterns, and build predictive models to improve sales performance.
\\
\\
\noindent \textbf{Features:}
\begin{itemize}
    \item Customer information (demographics, firmographics, etc.)
    \item Sales activities
    \item Opportunity data (deal size, stage, probability, etc.)
    \item Product/service information
    \item Sales team and performance metrics
    \item Time-series data (daily/weekly/monthly sales, etc.)
\end{itemize}

\noindent \textbf{Use Cases:}
\begin{itemize}
    \item Predicting won/lost opportunities
    \item Forecasting deal size
    \item Identifying key drivers of sales performance
    \item Optimizing sales team performance
    \item Analyzing customer behavior and preferences
\end{itemize}

This dataset is perfect for data scientists, analysts, and students looking to practice their skills in:
\begin{itemize}
    \item Predictive modeling
    \item Data visualization
    \item Sales analytics
    \item Customer relationship management
\end{itemize}

Get started: Download the dataset and start exploring!

\subsection{Dashboard Specifications}

\subsubsection{Dashboard Requirements}

\noindent \textbf{Key Metrics and KPIs:}
\begin{itemize}
    \item Total Revenue
    \item Deal Conversion Rate
    \item Average Deal Size
    \item Sales by Product
    \item Top Performing Sales Agents
    \item Sales Pipeline Status
    \item Win Rate
    \item Average Sales Cycle Length
    \item Revenue by Sector
    \item Customer Acquisition Cost (CAC)
    \item Customer Lifetime Value (CLV)
    \item Lead Response Time
    \item Deal Stage Duration
    \item Regional Sales Performance
    \item Sales Forecast Accuracy
\end{itemize}

\noindent \textbf{User Stories:}
\begin{itemize}
    \item As a Sales Manager, I want to see the total revenue generated by the sales team to assess overall performance.
    \item As a Business Executive, I want to analyze the deal conversion rate to understand the efficiency of the sales process.
    \item As a Potential Client, I want to see the top-performing products to understand the company’s product strengths.
    \item As a Regional Manager, I want to monitor the performance of sales agents in my region to provide targeted support and training.
    \item As a CEO, I want to see the win rate to understand the overall success rate of our sales efforts.
    \item As a Sales Director, I want to analyze the average sales cycle length to identify bottlenecks in the sales process.
    \item As a Marketing Manager, I want to understand the customer acquisition cost to evaluate the efficiency of our marketing campaigns.
    \item As a Financial Analyst, I want to calculate the customer lifetime value to help with financial forecasting and budgeting.
    \item As a Sales Trainer, I want to review the lead response time to improve training programs for quicker lead engagement.
    \item As a Regional Manager, I want to compare sales performance across different regions to identify high and low performing areas.
    \item As an Operations Manager, I want to track deal stage duration to optimize the sales process and reduce delays.
    \item As a Sales Analyst, I want to assess sales forecast accuracy to improve our sales planning and predictions.
    \item As a Business Development Manager, I want to see revenue by sector to target high-potential industries for growth.
\end{itemize}


\subsection{Structure of the Document}
This document is organized into several sections, each covering a specific dashboard:
\begin{itemize}
    \item Homepage Dashboard: Overview and key metrics.
    \item Revenue Dashboard: Detailed revenue metrics and trends.
    \item Product Performance Dashboard: Insights into product sales and performance.
    \item Sales Team Dashboard: Analysis of sales agent performance.
    \item Pipeline Analysis Dashboard: Visualization of the sales pipeline stages and performance.
\end{itemize}
\newpage

\section{Homepage Dashboard}
% Homepage Dashboard content goes here.

\section{Revenue Dashboard}
% Revenue Dashboard content goes here.

\section{Product Performance Dashboard}
% Product Performance Dashboard content goes here.

\section{Sales Team Dashboard}
% Sales Team Dashboard content goes here.

\section{Pipeline Analysis Dashboard}
% Pipeline Analysis Dashboard content goes here.

\end{document}
